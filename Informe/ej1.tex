\section{Celda Sallen-Key}
El objetivo de esta secci\'on es la construcci\'on de dos filtros que cumplan con un conjunto de especificaciones,
empleando para el dise\~no en cascada la celda Sallen-Key. Para esto \'ultimo, es necesario primero realizar un an\'alisis ideal
y real de dicha celda para obtener ciertas conclusiones que faciliten el proceso de dise\~no e impongan restricciones sobre
las exigencias que se definan a cada etapa. Es importante aclarar que s\'olo se analizar\'a la celda pasabajos de Sallen-Key dado
que es la \'unica necesaria para la realizaci\'on de los filtros en cuesti\'on.

\subsection{An\'alisis ideal}
En el siguiente an\'alisis se asume un comportamiento ideal del amplificador operacional, considerando una impedancia de entrada del mismo
$Z_{IN} \rightarrow \infty$, corrientes de entrada e impedancia de salida nulas, y finalmente una ganancia tambi\'en $A_{VOL} \rightarrow \infty$.

El an'alisis ideal consiste en determinar la funci\'on transferencia y los par\'ametros caracter\'isticos para configurar la celda, as\'i como tambi\'en un an\'alisis de las sensibilidades relativas
respecto de los componentes a utilizar y un conjunto de estrategias propuestas para poder realizar el ajuste correcto de la celda. Por otro lado, es de inter\'es analizar las impedancias de entrada y salida de la celda,
ya que las mismas son las que permiten determinar hasta qu\'e punto es v\'alido asumir que en la conexi\'on en cascada no se produce una carga de una etapa con otra y efectivamente se puede aproximar $H(s) = H_1(s) \cdot H_2(s) \cdot ... \cdot H_n(s)$.

\begin{figure}[H]
    \centering
    \includegraphics[scale=0.45]{../EJ1/Recursos/circuito_sallen_key_pasabajos.png}
    \caption{Celda Sallen-Key para pasabajos}
    \label{circuito_sallen_key_pasabajos}
\end{figure}

\subsubsection{Funci\'on transferencia y par\'ametros}
Dentro del marco de los criterios expuestos anteriormente, para hallar la funci\'on transferencia se define el par\'ametro de ganancia de la celda $K$, y luego
se plantean un conjunto de ecuaciones empleando las leyes de Kirchhoff. Finalmente, el sistema se resuelve y se obtiene la funci\'on $H(s)$ de la cual se deducen par\'ametros
caracter\'isticos como $\omega_o$ y $Q$.

\begin{equation}
    K = 1 + \frac{R_B}{R_A}
    \label{eq:sallen_key_k}
\end{equation}

\begin{align*}
    & V_o = V_b \cdot K \\
    & V_b = V_a \cdot \frac{\frac{1}{s \cdot C_2}}{\frac{1}{s \cdot C_2} \cdot R_2} \\
    & \frac{V_i - V_a}{R_1} = \frac{V_a - V_o}{\frac{1}{s \cdot C_1}} + \frac{V_a - V_b}{R_2}
\end{align*}

\begin{equation}
    H(s) = \frac{K}{C_1 \cdot C_2 \cdot R_1 \cdot R_2 \cdot s^{2} + s \cdot \left[ C_2 \cdot (R_1 + R_2) + C_1 \cdot R_1 \cdot (1 - K) \right] + 1}
    \label{eq:sallen_key_h}
\end{equation}

\begin{equation}
    \omega_o = \frac{1}{\sqrt{C_1 \cdot C_2 \cdot R_1 \cdot R_2}}
    \label{eq:sallen_key_wo}
\end{equation}

\begin{equation}
    Q = \frac{\sqrt{C_1 \cdot C_2 \cdot R_1 \cdot R_2}}{C_2 \cdot (R_1 + R_2) + C_1 \cdot R_1 \cdot (1 - K)}
    \label{eq:sallen_key_q}
\end{equation}

A partir de estos resultados se puede observar que el par\'ametro definido $K$ corresponde a la ganancia de la banda de paso, o en continua. Por otro lado, es importante
se\~nalar la influencia de tal ganancia, entre otras variables, sobre el valor de la selecitividad $Q$ del circuito. Esto \'ultimo deber\'a ser tenido en cuenta en el an\'alisis de sensibilidades.

\subsubsection{Sensibilidades}
En el siguiente an\'alisis se emplea la definici\'on de sensibilidades relativas para cada una de las magnitudes o par\'ametros caracter\'isticos de la funci\'on transferencia. Esto es,
calcular $S^{y}_x = \frac{x_o}{y(x_o)} \cdot \frac{\delta y}{\delta x}$.

\begin{table}[H]
    \centering
    \begin{tabular}{c c c c c c}
        $R_A$ & $R_B$ & $R_1$ & $R_2$ & $C_1$ & $C_2$ \\
        \hline \\
        $0$ & $0$ & $\frac{-1}{2}$ & $\frac{-1}{2}$ & $\frac{-1}{2}$ & $\frac{-1}{2}$ \\
        \\
        \hline
    \end{tabular}
    \caption{Sensibilidades de $\omega_o$}
\end{table}

\begin{table}[H]
    \centering
    \begin{tabular}{c c c c c c}
        $R_A$ & $R_B$ & $R_1$ & $R_2$ & $C_1$ & $C_2$ \\
        \hline \\
        $\frac{1}{K} - 1$ & $1 - \frac{1}{K}$ & $0$ & $0$ & $0$ & $0$ \\
        \\
        \hline
    \end{tabular}
    \caption{Sensibilidades de $K$}
\end{table}

\begin{table}[H]
    \centering
    \begin{tabular}{c | c}
        \hline \\
        $K$ & $K \cdot Q \cdot \sqrt{\frac{C_1 \cdot R_1}{C_2 \cdot R_2}}$ \\
        \\ \hline \\
        $R_A$ & $-Q \cdot \frac{R_B}{R_A} \cdot \sqrt{\frac{R_1 \cdot C_1}{R_2 \cdot C_2}}$ \\
        \\ \hline \\
        $R_B$ & $Q \cdot \frac{R_B}{R_A} \cdot \sqrt{\frac{R_1 \cdot C_1}{R_2 \cdot C_2}}$ \\
        \\ \hline \\
        $R_1$ & $\frac{-1}{2} + Q \cdot \sqrt{\frac{C_2 \cdot R_2}{C_1 \cdot R_1}}$ \\
        \\ \hline \\
        $R_2$ & $\frac{1}{2} - Q \cdot \sqrt{\frac{C_2 \cdot R_2}{C_1 \cdot R_1}}$ \\
        \\ \hline \\
        $C_1$ & $\frac{-1}{2} + Q \cdot \left[ \sqrt{\frac{C_1 \cdot R_1}{C_2 \cdot R_2}} + \sqrt{\frac{C_1 \cdot R_2}{C_2 \cdot R_1}} \right]$ \\
        \\ \hline \\
        $C_2$ & $\frac{1}{2} - Q \cdot \left[ \sqrt{\frac{C_1 \cdot R_1}{C_2 \cdot R_2}} + \sqrt{\frac{C_1 \cdot R_2}{C_2 \cdot R_1}} \right]$ \\
        \\ \hline \\
    \end{tabular}
    \caption{Sensibilidades de $Q$}
\end{table}

\subsubsection{M\'etodos de ajuste}
En el proceso de dise\~no en cascada, para la composici\'on de cada una de las etapas se realiza una agrupaci\'on de ceros y polos acordes para la formaci\'on de una 
funci\'on transferencia que pueda ser implementada por alguna de las celdas existentes o disponibles, no obstante su implementaci\'on requiere de un proceso de ajuste para garantizar
el dise\~no menos sensible a variaciones por imperfecciones de los componentes o efectos par\'asitos de los mismos. En este apartado se proponen y analizan diferentes m\'etodos o enfoques de dise\~no, 
determinando sus beneficios al momento de ajustar los par\'ametros de la celda a la etapa deseada.

\paragraph{Dise\~no por componentes iguales:} Como estrategia de dise\~no para facilitar la elecci\'on de componentes ante los grados de libertad de los cuales se disponen,
se propone $R = R_1 = R_2$ y luego $C = C_1 = C_2$. De esta imposici\'on se simplifican las expresiones de los par\'ametros y de las sensibilidades, obteniendo:

\begin{equation}
    \omega_o = \frac{1}{R \cdot C}
    \label{eq:wo_ajuste_componentes_iguales}
\end{equation}

\begin{equation}
    Q = \frac{1}{3 - K}
    \label{eq:q_ajuste_componentes_iguales}
\end{equation}

\begin{table}[H]
    \centering
    \begin{tabular}{c | c}
        \hline \\
        $K$ & $\frac{K}{3 - K}$ \\
        \\ \hline \\
        $R_A$ & $-Q \cdot \frac{R_B}{R_A}$ \\
        \\ \hline \\
        $R_B$ & $Q \cdot \frac{R_B}{R_A}$ \\
        \\ \hline \\
        $R_1$ & $\frac{-1}{2} + Q$ \\
        \\ \hline \\
        $R_2$ & $\frac{1}{2} - Q$ \\
        \\ \hline \\
        $C_1$ & $\frac{-1}{2} + Q \cdot 2$ \\
        \\ \hline \\
        $C_2$ & $\frac{1}{2} - Q \cdot 2$ \\
        \\ \hline \\
    \end{tabular}
    \caption{Sensibilidades de $Q$}
\end{table}

Estos resultados revelan que si bien es m\'as simple la elecci\'on de componentes, el ajuste de la celda se vuelve m\'as complejo cuando la etapa requiere una selectividad alta,
provocando que ante variaciones de los componentes, para sistemas de selectividad muy alta, cambie el comportamiento del sistema dr\'asticamente. No obstante, este enfoque puede simplificar
el proceso de dise\~no en los casos donde se requiera un Q bajo.

\paragraph{Dise\~no por componentes proporcionales:} Como estrategia se impone que no haya ganancia de banda de paso en esta etapa, es decir que $K = 1$, y luego que los componentes sean proporcionales
entre s\'i, para lo cual se definen dos constantes de proporcionalidad tomando como referencia $R_2 = R \Rightarrow R_1 = m \cdot R$ y $C_2 = C \Rightarrow C_1 = n \cdot C$.

\begin{equation}
    \omega_o = \frac{1}{R \cdot C \cdot \sqrt{m \cdot n}}
    \label{eq:wo_ajuste_componentes_propocionales}
\end{equation}

\begin{equation}
    Q = \frac{\sqrt{m \cdot n}}{1 + m}
    \label{eq:q_ajuste_componentes_proporcionales}
\end{equation}


\begin{table}[H]
    \centering
    \begin{tabular}{c | c}
        \hline \\
        $K$ & $\frac{m \cdot n}{1 + m}$ \\
        \\ \hline \\
        $R_A$ & $\frac{m \cdot n}{1 + m} \cdot \frac{R_B}{R_A}$ \\
        \\ \hline \\
        $R_B$ & $- \frac{m \cdot n}{1 + m} \cdot \frac{R_B}{R_A}$ \\
        \\ \hline \\
        $R_1$ & $\frac{-1}{2} + \frac{1}{1 + m}$ \\
        \\ \hline \\
        $R_2$ & $\frac{1}{2} - \frac{1}{1 + m}$ \\
        \\ \hline \\
        $C_1$ & $\frac{-1}{2} + 2 \cdot \frac{m \cdot n}{1 + m}$ \\
        \\ \hline \\
        $C_2$ & $\frac{1}{2} - 2 \cdot \frac{m \cdot n}{1 + m}$ \\
        \\ \hline \\
    \end{tabular}
    \caption{Sensibilidades de $Q$}
\end{table}

En primer lugar uno de los beneficios de esta estrategia es poder imponer restricciones adicionales sobre el dise\~no, como asignar valores a $m$ o a $n$ por separado. Esto \'ultimo permitir\'ia seg\'un
el caso minimizar la sensibilidad, suponiendo por ejemplo que $m = 1$, o en otro caso que $n = \frac{1}{2}$. Es importante aclarar que seg\'un el enfoque aplicado, los valores de referencia $R$ y $C$ son utilizados
para determinar el valor de la frecuencia de corte del filtro, no obstante las proporcionalidades permiten ajustar adem\'as la selectividad, con lo cual s\'olo uno de los dos criterios para minimizar la sensibilidad puede ser utilizado.
En segundo lugar, se puede observar que a diferencia de antes la estabilidad frente a variaciones de los valores nominales de componentes es mayor, no obstante esto implica una cota superior a la selectividad que puede garantizar esta celda,
para el caso donde $m = 1$ y luego, $Q = \frac{\sqrt{n}}{2}$.

En conclusi\'on, este enfoque de dise\~no permite seg\'un se tenga mayor probabilidad de variaciones en los capacitores o resistencias, ajustar las sensibilidades para minimizar el efecto de tales componentes,
y adem\'as definir los valores de forma m\'as simplificada. Adem\'as, al buscar que $K = 1$, se reduce el amplificador a un seguidor de tensi\'on o buffer, lo cual tiene sus ventajas al reducir los componentes necesitados
y aumentar el ancho de banda de tal etapa.

\paragraph{Dise\~no con atenuaci\'on:} Es posible que durante el uso de etapas se necesite uan determinada atenuaci\'on en la misma si se considera que la entrada puede ser de alta se\~nal, volviendose susceptible a saturar
y distorsionar. Esto \'ultimo, como se ver\'a posteriormente, afectar\'a directamente al rango din\'anmico de la celda dise\~nada. Por lo tanto, es deseable poder imponer una atenuaci\'on que mejore tales caracter\'isticas, pero en la
configuraci\'on propuesta no se puede conseguir un $K < 1$, salvo que se utilicen resistencias negativas con alg\'un circuito adicional pero eso agregar\'ia complejidad. Entonces, para solucionarlo se propone agregar una resistencia que produzca
una divisi\'on de tensi\'on, de forma tal que al aplicar el teorema de Thevenin, todo el an\'alisis previo es replicable asumiendo que la $R_1 = R_{TH} = R_{1_A} // R_{1_B}$.

\begin{figure}[H]
    \centering
    \includegraphics[scale=0.5]{../EJ1/Recursos/circuito_sallen_key_pasabajos_atenuado.png}
    \caption{Circuito Sallen-Key pasabajos con atenuaci\'on}
    \label{fig:sallen_key_pasabajos_atenuado}
\end{figure}

\begin{equation}
    A = \frac{R_{1_B}}{R_{1_A} + R_{1_B}}
\end{equation}

\begin{equation}
    H(s) = \frac{K \cdot A}{C_1 \cdot C_2 \cdot R_1 \cdot R_2 \cdot s^{2} + s \cdot \left[ C_2 \cdot (R_1 + R_2) + C_1 \cdot R_1 \cdot (1 - K) \right] + 1}
    \label{eq:sallen_key_h_atenuada}
\end{equation}

\subsubsection{Impedancia de entrada}
En la Fig. \ref{fig:circuito_sallen_key_impedancia_entrada_ideal} se puede observar que para calcular la impedancia de entrada se debe plantear un sistema de ecuaciones empleando 
las leyes de Kirchhoff correspondientes.

\begin{figure}[H]
    \centering
    \includegraphics[scale=0.5]{../EJ1/Recursos/circuito_sallen_key_impedancia.png}
    \caption{Circuito Sallen-Key pasabajos para c\'alculo de $Z_{IN}(s)$}
    \label{fig:circuito_sallen_key_impedancia_entrada_ideal}
\end{figure}

\begin{align*}
    & \frac{V_i - V_A}{R_1} = \frac{V_A}{R_2 + \frac{1}{s \cdot C_2}} + \frac{V_A - V_o}{\frac{1}{s \cdot C_2}} \\
    & V_o = V_A \cdot K \cdot \frac{\frac{1}{s \cdot C_2}}{R_2 + \frac{1}{s \cdot C_2}} \\
    & I_2 = \frac{V_A}{R_2 + \frac{1}{s \cdot C_2}} \\
    & I_1 = \frac{V_A - V_o}{\frac{1}{s \cdot C_1}}
\end{align*}

\begin{equation}
    Z_{IN}(s) = \frac{C_1 \cdot C_2 \cdot R_1 \cdot R_2 \cdot s^{2} + s \cdot \left[ C_2 \cdot (R_1 + R_2) + C_1 \cdot R_1 \cdot (1 - K) \right] + 1}{s \cdot \left[C_2 + C_1 \cdot (1 - K) \right] \cdot \left[ 1 + \frac{s \cdot C_1 \cdot C_2 \cdot R_2}{C_2 + C_1 \cdot (1 - K)} \right]}
    \label{eq:sallen_key_impedancia_entrada_ideal}
\end{equation}

A partir de la expresi\'on de la Ec. \ref{eq:sallen_key_impedancia_entrada_ideal}, en primer lugar se puede observar que la etapa como amplificador de transimpedancia es inestable y que para corriente continua la impedancia de entrada $Z_{IN} \rightarrow \infty$.
No obstante, es de inter\'es realizar un an\'alisis sobre esta expresi\'on que permita estimar de cierta forma una condici\'on de dise\~no para garantizar que la conexi\'on en cascada de la etapa es viable sin producir mayores desviaciones en la funci\'on transferencia total.


Se propone observar que la funci\'on $Z_{IN}(s)$ es de segundo orden, posee un polo en el origen y uno real, y luego dos ceros con frecuencia de corte ubicada en donde se encuentra la del filtro pasabajos que constituye esta celda. Asumiendo que el sistema se encuentra en todo caso cr\'iticamente amortiguado o subamortiguado y luego empleando las aproximaciones asint\'oticas
utilizadas en un diagrama de bode, se puede asumir que para las frecuencias de la banda de paso de este filtro, los ceros no tienen mayor influencia. As\'i, se reduce la complejidad del an\'alisis ya que cuando se vuelva apreciable el efecto que estas tienen, se estar\'a en la banda de transici\'on o de rechazo.
De esta forma, sea que el polo real se encuentra antes o despu\'es de dicha frecuencia de corte, para la banda de paso la impedancia de entrada es mon\'otona decreciente, con lo cual podemos estimar de manera aproximada que dentro de esta regi\'on de inter\'es el valor m\'inimo se encuentra sobre la frecuencia de corte,
con lo cual $Z_{IN_{min}} \approx Z_{IN}(s=j \cdot \omega_o)$. En la Fig. \ref{fig:grafico_zin_ideal_arbitraria} se observa un gr\'afico con valores arbitrarios para apreciar lo explicado.

\begin{figure}[H]
    \centering
    \includegraphics[scale=0.6]{../EJ1/Recursos/impedancia_entrada_arbitraria.png}
    \caption{Gr\'afico de $|Z_{IN}(j \cdot \omega)|$ con valores arbitrarios}
    \label{fig:grafico_zin_ideal_arbitraria}
\end{figure}

Entonces, reutilizando las expresiones presentadas y usando la definici\'on de la frecuencia de corte como se la encontr\'o en apartados anteriores, se llega a la siguiente expresi\'on de la impedancia de entrada
m\'inima en toda la banda de paso.

\begin{equation}
    Z_{IN_{min}} = \frac{C_2 \cdot (R_1 + R_2) + C_1 \cdot R_1 \cdot (1 - K)}{C_1 \cdot (1 - K) + C_2 + j \cdot \sqrt{\frac{C_1 \cdot C_2 \cdot R_2}{R_1}}}
\end{equation}

Puesto que se han propuesto m\'etodos de ajuste apropiados para la celda estudiada, se simplifica esta expresi\'on en el marco de tales enfoques, para obtener una expresi\'on
reducido en funci\'on de sus respectivos coeficientes o par\'ametros.

Para el dise\~no con componentes iguales:

\begin{equation}
    |Z_{IN_{min}}| = \frac{R \cdot |3 - K|}{\sqrt{K^{2} - 4 \cdot K + 5}}
    \label{eq:impedancia_minima_componentes_iguales}
\end{equation}

Para el dise\~no con componentes proporcionales:

\begin{equation}
    |Z_{IN_{min}}| = \frac{R \cdot (1 + m)}{\sqrt{1 + \frac{n}{m}}}
    \label{eq:impedancia_minima_componentes_proporcionales}
\end{equation}

\begin{figure}[H]
    \centering
    \includegraphics[scale=0.45]{../EJ1/Recursos/impedancia_entrada_componentes_iguales.png}
    \caption{Impedancia de entrada m\'inima con componentes iguales}
\end{figure}

\begin{figure}[H]
    \centering
    \includegraphics[scale=0.7]{../EJ1/Recursos/impedancia_entrada_componentes_proporcionales.png}
    \caption{Impedancia de entrada m\'inima con componentes proporcionales}
\end{figure}

\subsubsection{Impedancia de salida}
En el contexto de an\'alisis ideal, dado que la salida del circuito est\'a impuesta por un amplificador operacional con realimentaci\'on negativa,
luego la impedancia de salida tiende a ser nula. Es decir, $Z_O = 0$ siempre y cuando sea considerado un amplificador operacional ideal.

\subsubsection{Rango din\'amico}

\subsection{An\'alisis real}

\subsubsection{Funci\'on transferencia}

\subsubsection{Impedancia de entrada}x

\subsubsection{Impedancia de salida}

\subsection{Dise\~no de un filtro con Legendre}

\subsubsection{Especificaciones y funci\'on aproximaci\'on}

\subsubsection{Dise\~no de etapas}

\subsubsection{Simulaci\'on y verificaci\'on}

\subsubsection{Dise\~no de PCB}

\subsubsection{Resultados pr\'acticos}

\subsection{Dse\~no de un filtro con Bessel}

\subsubsection{Especificaciones y funci\'on aproximaci\'on}

\subsubsection{Dise\~no de etapas}

\subsubsection{Simulaci\'on y verificaci\'on}

\subsubsection{Dise\~no de PCB}

\subsubsection{Resultados pr\'acticos}

\subsection{Conclusiones}